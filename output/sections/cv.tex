\section{学術論文(査読有)}
\begin{enumerate}[label={[\thesection.\arabic*]}]
\item Daijiro Mizutani, Rie Ikushima, Koki Satsukawa, Yosuke Kawasaki and Masao Kuwahara, Integrating real-time monitoring information into asset failure modeling: application to ETC facilities, \textit{Journal of Infrastructure Systems}, 2025 (Accepted).
\item Koki Satsukawa, Takamasa Iryo, Naoki Yoshizawa, Michael J. Smith and David Watling, Adjustment process of adaptive signal control strategies with route choices: a case study with Policy P0, \textit{Transportmetrica B: Transport Dynamics}, 13(1), 2025.
\item Daijiro Mizutani, Shunichi Fukuyama and Koki Satsukawa, Optimal road facility spare parts location with continuum approximation, \textit{Transportation Research Part C: Emerging Technologies}, 174, 105109, 2025.
\item Koki Satsukawa, Kentaro Wada and Takamasa Iryo, Stability analysis of a departure time choice problem with atomic vehicle models, \textit{Transportation Research Part B: Methodological}, 189, 103039, 2024.
\item Takara Sakai, Takashi Akamatsu and Koki Satsukawa, Queue replacement principle for corridor problems with heterogeneous commuters, \textit{Transportation Research Part B: Methodological}, 187, 103024, 2024.
\item Takara Sakai, Takashi Akamatsu and Koki Satsukawa, A paradox of telecommuting and staggered work hours in the bottleneck model, \textit{Transportation Science}, 58(6), 1335-1351, 2024.
\item 佐津川功季, 原祐輔, 川崎洋輔, 井料隆雅, 需要の不確実性に対する都市交通サービスのバンドル予約システムの設計, \textit{土木学会論文集}, 80(9), 23-00206, 2024.
\item Daijiro Mizutani, Yuto Nakazato, Rie Ikushima, Koki Satsukawa, Yosuke Kawasaki and Masao Kuwahara, Optimal intervention policy of emergency storage batteries for expressway facilities considering deterioration risk during lead time of replacement, \textit{Reliability Engineering \& System Safety}, 242, 109735, 2024.
\item 原祐輔, 川崎洋輔, 佐津川功季, 井料隆雅, 予約システムにおける探索行動と選好誘出の影響評価のための実験的アプローチ, \textit{土木学会論文集}, 79(20), 23-20020, 2023.
\item 佐津川功季, 水谷大二郎, 川崎洋輔, 金田威夫, 桑原雅夫, 故障時交通渋滞による経済損失を考慮したETC設備の最適補修施策に関する研究, \textit{土木学会論文集D3(土木計画学)}, 78(3), 105-121, 2022.
\item Haoran Fu, Takashi Akamatsu, Koki Satsukawa and Kentaro Wada, Dynamic traffic assignment in a corridor network with multiple bottlenecks: optimum versus equilibrium, \textit{Transportation Research Part B: Methodological}, 161, 218-246, 2022.
\item Michael J. Smith, Takamasa Iryo, Richard Mounce, Koki Satsukawa and David Watling, Zero-queue traffic control, using green-times and prices together, \textit{Transportation Research Part C: Emerging Technologies}, 138, 103630, 2022.
\item 生嶋理恵, 水谷大二郎, 佐津川功季, 川崎洋輔, 桑原雅夫, 現場技術者へのアンケート調査に基づく高速道路設備の維持管理施策の改善可能性, \textit{土木学会論文集F4(建設マネジメント)}, 78(1), 51-69, 2022.
\item Koki Satsukawa, Kentaro Wada and David Watling, Dynamic system optimal assignment with atomic users: convergence and stability, \textit{Transportation Research Part B: Methodological}, 155, 188-209, 2022.
\item 酒井高良, 赤松隆, 佐津川功季, スケジュールコストの異質性を考慮したタンデムボトルネック出発時刻選択問題, \textit{土木学会論文集D3(土木計画学)}, 77(4), 330-345, 2021.
\item Takashi Akamatsu, Takeshi Nagae, Minoru Osawa, Koki Satsukawa, Takara Sakai and Daijiro Mizutani, Model-based analysis on social acceptability and feasibility of a focused protection strategy against the COVID-19 pandemic, \textit{Scientific Reports}, 11, 2003, 2021.
\item 水谷大二郎, 川崎洋輔, 佐津川功季, 中川岳士, 梅田祥吾, 生嶋理恵, 桑原雅夫, 利用者の経済損失を考慮した高速道路情報板の維持管理施策の簡易的評価手法, \textit{土木学会論文集D3(土木計画学)}, 76(5), I\_127-I\_139, 2021.
\item Koki Satsukawa, Kentaro Wada and Takamasa Iryo, Stochastic stability of dynamic user equilibrium in unidirectional networks: weakly acyclic game approach, \textit{Transportation Research Part B: Methodological}, 125, 229-247, 2019.
\item Kentaro Wada, Koki Satsukawa, Mike Smith and Takashi Akamatsu, Network throughput under dynamic user equilibrium: queue spillback, paradox and traffic control, \textit{Transportation Research Part B: Methodological}, 126, 391-413, 2019.
\item 和田健太郎, 瀬尾亨, 中西航, 佐津川功季, 柳原正実, Kinematic Wave理論の近年の発展:変分理論とネットワーク拡張, \textit{土木学会論文集D3(土木計画学)}, 73(5), I\_1139-I\_1158, 2017.
\item 和田健太郎, 佐津川功季, 動的配分理論による道路ネットワークの交通性能解析, \textit{土木学会論文集D3(土木計画学)}, 73(1), 56-72, 2017.
\item 佐津川功季, 和田健太郎, 単一終点ネットワークにおける動的交通量配分問題のNash均衡解の解法について, \textit{土木学会論文集D3(土木計画学)}, 73(1), 103-108, 2017.
\end{enumerate}

\section{国際会議(査読有)}
\begin{enumerate}[label={[\thesection.\arabic*]}]
\item Koki Satsukawa and Yuki Takayama, A bottleneck model with shared autonomous vehicles: Scale economies and price regulations, \textit{The 26th International Symposium on Transportation and Traffic Theory (ISTTT26)}, Munich, Germany, July 2026 (Accepted).
\item Riki Kawase, Koki Satsukawa and Toru Seo, Flexible and reliable transportation network design for emerging transportation services: multi-stage stochastic programming approach, \textit{The 26th International Symposium on Transportation and Traffic Theory (ISTTT26)}, Munich, Germany, July 2026 (Accepted).
\item Takara Sakai, Takashi Akamatsu and Koki Satsukawa, Queue replacement approach to dynamic user equilibrium assignment with route and departure time choice, \textit{The 26th International Symposium on Transportation and Traffic Theory (ISTTT26)}, Munich, Germany, July 2026 (Accepted).
\item Sowa Suzuki, Haruki Takiguchi, Takamasa Iryo, Haruko Nakao, Koki Satsukawa, David Watling and Richard Connors, Designing pricing strategies for community-owned shared transport: An evolutionary approach, \textit{The 29th International Conference of Hong Kong Society for Transportation Studies (29th HKSTS)}, Hong Kong, China, December 2025.
\item Takeru Nihei, Takamasa Iryo and Koki Satsukawa, A distributed evolutionary algorithm for optimisation of public transport operations with mixed-capacity vehicle fleets, \textit{The 29th International Conference of Hong Kong Society for Transportation Studies (29th HKSTS)}, Hong Kong, China, December 2025.
\item Michael Smith, David Watling, Ronghui Liu, Koki Satsukawa, Takamasa Iryo and Richard Mounce, New stable responsive local gating strategies to control vehicle queues and flows in congested urban networks, \textit{The 10th International Symposium on Dynamic Traffic Assignment (DTA2025)}, Salerno, Italy, September 2025.
\item Takamasa Iryo, David Watling, Koki Satsukawa, Richard Connors, Haruko Nakao and Sowa Suzuki, Markov-chain-based model for the evolutionary process of self-financed shared mobility systems: Theoretical assessments, \textit{The 10th International Symposium on Dynamic Traffic Assignment (DTA2025)}, Salerno, Italy, September 2025.
\item Haruki Takiguchi, Sowa Suzuki, Takamasa Iryo, Koki Satsukawa and Haruko Nakao, Assessing the evolutionary process of self-financed shared mobility systems: A laboratory experimental approach, \textit{The 10th International Symposium on Dynamic Traffic Assignment (DTA2025)}, Salerno, Italy, September 2025.
\item Yuki Kosaka, Shota Tsurimoto, Masahiro Noguchi, Koki Satsukawa, Masayuki Takamura and Akihiro Nomura, Exploring optimal lifestyle modification pathway for preventing Cardiovascular disease using machine learning and pathfinding algorithms, \textit{European Society of Cardiology Congress 2025 (ESC2025)}, Madrid, Spain, August 2025.
\item Haruko Nakao, Koki Satsukawa, Takamasa Iryo, Richard Connors and Sowa Suzuki, Evolutionary process of self-financed shared mobility systems, \textit{The Twelfth Triennial Symposium on Transportation Analysis (TRISTAN XII)}, Okinawa, Japan, June 2025.
\item Koki Satsukawa, Kentaro Wada and Takamasa Iryo, Stability analysis of a departure time choice problem with atomic vehicle models, \textit{The 25th International Symposium on Transportation and Traffic Theory (ISTTT25)}, Ann Arbor, USA, July 2024.
\item Takara Sakai, Takashi Akamatsu and Koki Satsukawa, A paradox of telecommuting and staggered work hours in the bottleneck model, \textit{The 25th International Symposium on Transportation and Traffic Theory (ISTTT25)}, Ann Arbor, USA, July 2024.
\item Koki Satsukawa, Yusuke Hara, Yosuke Kawasaki and Takamasa Iryo, A study on the design of a reservation system for urban transport services under uncertainty, \textit{The 9th International Symposium on Transport Network Resilience (INSTR2023)}, Hong Kong, China, December 2023.
\item Takara Sakai, Takashi Akamatsu and Koki Satsukawa, Welfare impacts of remote and flexible working policies in the bottleneck model, \textit{11th Symposium of the European Association for Research in Transportation (hEART2023)}, Zurich, Switzerland, September 2023.
\item Koki Satsukawa, Takamasa Iryo, Naoki Yoshizawa, Michael J. Smith and David Watling, Adjustment process of adaptive signal control strategies with route choices: a case study with Policy P0, \textit{9th International Symposium on Dynamic Traffic Assignment (DTA2023)}, Chicago, USA, July 2023.
\item Takara Sakai, Takashi Akamatsu and Koki Satsukawa, Queue replacement principle for corridor problems with heterogeneous commuters, \textit{9th International Symposium on Dynamic Traffic Assignment (DTA2023)}, Chicago, USA, July 2023.
\item Koki Satsukawa, Kentaro Wada and David Watling, Dynamic system optimal traffic assignment with atomic users: Convergence and stability, \textit{The 24th International Symposium on Transportation and Traffic Theory (ISTTT24)}, Beijing, China (online), July 2022.
\item Takara Sakai, Takashi Akamatsu and Koki Satsukawa, Departure time choice problems in a corridor network with heterogeneous value of schedule delay, \textit{The 25th International Conference of Hong Kong Society for Transportation Studies (25th HKSTS)}, Hong Kong, China (online), December 2021.
\item Michael J. Smith, Takamasa Iryo, Richard Mounce, Koki Satsukawa and David Watling, Zero-queue traffic control using green-times and prices together, \textit{The 8th International Symposium on Dynamic Traffic Assignment}, Seattle, USA (online), June 2021.
\item Koki Satsukawa, Kentaro Wada and Takamasa Iryo, Stochastic stability of dynamic user equilibrium in unidirectional networks: Weakly acyclic game approach, \textit{The 23rd International Symposium on Transportation and Traffic Theory (ISTTT23)}, Lausanne, Switzerland, July 2019.
\item Kentaro Wada and Koki Satsukawa, A theoretical analysis of Macroscopic Fundamental Diagram based on dynamic user equilibrium, \textit{The 6th International Symposium on Dynamic Traffic Assignment}, Sydney, Australia, June 2016.
\item Koki Satsukawa and Kentaro Wada, Effect of origin-destination structures on network performance: Some simple examples, \textit{The 20th International Conference of Hong Kong Society for Transportation Studies (20th HKSTS)}, Hong Kong, China, December 2015.
\end{enumerate}

\section{寄稿・解説}
\begin{enumerate}[label={[\thesection.\arabic*]}]
\item 佐津川功季, 和田健太郎, 渋滞の空間分布に基づく道路ネットワークの交通性能の解析理論, 生産研究, 71(2), 89-95, 2019.
\item 中西航, 佐津川功季, Kinematic Wave理論のネットワーク拡張, 交通工学, 52(4), 33-38, 2017.
\end{enumerate}

\section{招待講演}
\begin{enumerate}[label={[\thesection.\arabic*]}]
\item 佐津川功季, 和田健太郎, 動的ネットワーク交通流解析と確率進化ゲーム理論, 第33回RAMP数理最適化シンポジウム (The Thirty-Third RAMP Symposium), 南山大学, November 2021.
\item Koki Satsukawa, Network throughput under dynamic user equilibrium: queue spillback, paradox and traffic control, Frontier of the MFD study (1), Ehime University, Japan, June 2017.
\end{enumerate}

\section{その他会議発表等}
\subsection*{国際セミナー・ワークショップ}
\begin{enumerate}[label={[\thesection.\arabic*]}]
\item Koki Satsukawa, Convergence and stability analysis of dynamic traffic assignment with atomic games, Dagstuhl Seminar 24281, Dynamic Traffic Models in Transportation Science, July 2024.
\item Koki Satsukawa, Yusuke Hara, Yosuke Kawasaki and Takamasa Iryo, A study on the design of a flexible reservation system for urban transport services under uncertainty, International workshop on methodologies towards sustainable and flexible city transport systems, Tohoku University (Sendai), Japan, February 2024.
\item Koki Satsukawa and David Watling, Modelling and control of dynamic traffic flow in mixed networks of autonomous and human-driven vehicles, International Workshop on Control and DTA: from principles to large-scale implementation, University of Leeds (Leeds), UK, February 2020.
\item Koki Satsukawa, Network throughput under dynamic user equilibrium, Joint Seminar between Tongji University and the University of Tokyo and 13th International Workshop on ITS, The University of Tokyo, Tokyo, Japan, November 2018.
\item Koki Satsukawa, A comparative study of coordinated ramp metering strategies for complex networks, The 3rd Workshop on Intelligent Transport Systems between Tongji University and the University of Tokyo, Tongji University (Shanghai), China, October 2017.
\item Koki Satsukawa, Traffic performance analysis of road network based on dynamic user equilibrium, The 2nd workshop on ITS between Tongji University and the University of Tokyo, University of Tokyo (Tokyo), Japan, November 2016.
\item Koki Satsukawa, Kentaro Wada, An analysis of network throughputs for different origin-destination patterns, 7th Joint student seminar on civil infrastructure, Bangkok, Thailand, August 2015.
\end{enumerate}
\subsection*{国内会議(査読無)}
\begin{enumerate}[resume,label={[\thesection.\arabic*]}]
\item Tithmesa Chhor, 羽生昇平, 佐津川功季, 井料隆雅, 共有右折車線を有する交差点における強化学習に基づく交通信号制御の頑健性の評価, 第23回ITSシンポジウム, 広島国際会議場, December 2025.
\item 高山雄貴, 佐津川功季, A bottleneck model with shared autonomous vehicles: Scale economies and price regulations, 第39回応用地域学会 (ARSC) 研究発表大会, 富山大学, November 2025.
\item 河瀬理貴, 佐津川功季, 瀬尾亨, 不確実性下の動的システム最適配分の理論解析:確率計画法アプローチ, 土木計画学研究・講演集, vol.71, 香川大学, June 2025.
\item 酒井高良, 赤松隆, 佐津川功季, Queue replacement approach to dynamic user equilibrium assignment with route and departure time choice, 土木計画学研究・講演集, vol.71, 香川大学, June 2025.
\item 水谷大二郎, 桑原育也, 幸地佳, 深水貴斗, 佐津川功季, 川崎洋輔, 桑原雅夫, 常時モニタリング情報を用いたETC施設の故障予測, JAAM研究・実践発表会, November 2024.
\item 川崎洋輔, 水谷大二郎, 佐津川功季, 桑原雅夫, 緒方達也, 倉森健, 多次元センサーデータを用いた走行車両重量測定装置の異常検知手法の提案, 土木計画学研究・講演集, vol.70, 岡山大学, November 2024.
\item 原祐輔, 川崎洋輔, 佐津川功季, 井料隆雅, 予約システムにおける探索行動と選好誘出の影響評価のための実験的アプローチ, 土木計画学研究・講演集, vol.67, 福岡大学, June 2023.
\item 佐津川功季, 原祐輔, 川崎洋輔, 井料隆雅, 不確実性下における都市交通サービスの予約システム設計に関する研究, 土木計画学研究・講演集, vol.67, 福岡大学, June 2023.
\item 生嶋理恵, 水谷大二郎, 佐津川功季, 川崎洋輔, 桑原雅夫, 点検データの統計分析に基づく購読道路蓄電池設備の更新施策, JAAM研究・実践発表会, November 2022.
\item 福山峻一, 水谷大二郎, 佐津川功季, 連続体近似によるETC施設予備部品の最適配置計画, 土木計画学研究・講演集, vol.66, 琉球大学, November 2022.
\item 吉澤尚輝, 佐津川功季, 井料隆雅, 経路選択を考慮した適応型信号制御の動学解析, 土木計画学研究・講演集, vol.66, 琉球大学, November 2022.
\item 酒井高良, 赤松隆, 佐津川功季, 動的システム最適配分の大域的最適解:待ち行列は存在しうるか, 土木計画学研究・講演集, vol.65, 広島大学(オンライン), June 2022.
\item 吉澤尚輝, 佐津川功季, 井料隆雅, 経路選択と時間価値の異質性を考慮した自律分散型信号制御の評価, 土木計画学研究・講演集, vol.64, 福島大学(オンライン), December 2021.
\item 佐津川功季, 原祐輔, 川崎洋輔, 井料隆雅, 複数選択肢をバンドルする予約システムによる効率的な資源配分の理論解析, 土木計画学研究・講演集, vol.64, 福島大学(オンライン), December 2021.
\item 生嶋理恵, 水谷大二郎, 佐津川功季, 川崎洋輔, 桑原雅夫, 現場技術者へのアンケート調査に基づく高速道路設備の維持管理施策の改善可能性, 土木計画学研究・講演集, vol.64, 福島大学(オンライン), December 2021.
\item 酒井高良, 赤松隆, 佐津川功季, 利用者の異質性を考慮したコリドーネットワーク出発時刻選択問題, 第35回応用地域学会 (ARSC) 研究発表大会, 金沢大学(オンライン), November 2021.
\item 生嶋理恵, 水谷大二郎, 佐津川功季, 川崎洋輔, 桑原雅夫, 現場技術者の知見を活用した高速道路設備の維持管理施策の改善可能性, JAAM研究・実践発表会(オンライン), November 2021.
\item Yang Liu, Takashi Nagae and Koki Satsukawa, Dynamic user equilibrium in many-to-one corridor network with irregular ordered capacity patterns, 土木計画学研究・講演集, vol.63, 東北大学(オンライン), June 2021.
\item 酒井高良, 赤松隆, 佐津川功季, スケジュールコストの異質性を考慮したタンデムボトルネック出発時刻選択問題, 土木計画学研究・講演集, vol.63, 東北大学(オンライン), June 2021.
\item 川崎洋輔, 佐津川功季, 梅田祥吾, 桑原雅夫, 観光地における長期交通状態予測手法の提案, 第18回ITSシンポジウム, 愛媛大学(オンライン), December 2020.
\item 佐津川功季, 和田健太郎, ポテンシャル・ゲームに基づく動的システム最適配分の確率的安定性解析, 土木計画学研究・講演集, vol.62, 信州大学(オンライン), November 2020.
\item 佐津川功季, 水谷大二郎, 川崎洋輔, 金田威夫, 桑原雅夫, 高速道路施設を構成する部品備蓄の最適計画に関する研究, 土木計画学研究・講演集, vol.62, 信州大学(オンライン), November 2020.
\item 佐津川功季, 和田健太郎, 大口敬, 渋滞パターンに基づく道路ネットワークの交通性能とその低下メカニズムの解析, 東京大学空間情報科学研究センター全国共同利用研究発表大会「CSIS DAYS 2018」, 東京大学, November 2018.
\item 佐津川功季, 和田健太郎, 井料隆雅, ゲーム理論的アプローチによる動的利用者均衡の確率的安定性の解析, 土木計画学研究・講演集, vol.58, 大分大学, November 2018.
\item 佐津川功季, 和田健太郎, 大口敬, 渋滞パターンの縮約に基づく一般構造ネットワークの交通性能近似手法の考察, 土木計画学研究・講演集, vol.57, 東京工業大学, June 2018.
\item 佐津川功季, 和田健太郎, 大口敬, 複雑なネットワークにおけるMFDを活用した協調ランプ制御手法の有効性評価と考察, 第9回高速道路の交通データ利用に関する勉強会, 高知工科大学, October 2017.
\item 佐津川功季, 森部伸一, 和田健太郎, 大口敬, 首都圏高速道路ネットワークの効率的利用のためのランプ制御, 土木計画学研究・講演集, vol.55, 愛媛大学, June 2017.
\item 和田健太郎, 瀬尾亨, 中西航, 柳原正実, 佐津川功季, Kinematic Wave理論の近年の展開:交通流の変分理論とネットワーク拡張, 土木計画学研究・講演集, vol.54, P24, 長崎大学, November 2016.
\item 佐津川功季, 和田健太郎, 単一終点ネットワークにおける動的交通量配分問題のNash均衡解の解法について, 土木計画学研究・講演集, vol.54, 13, 長崎大学, November 2016.
\item 和田健太郎, 佐津川功季, 動的配分理論による道路ネットワークの交通性能解析, 土木計画学研究・講演集, vol.54, 12, 長崎大学, November 2016.
\item 佐津川功季, 和田健太郎, 渋滞パターンに着目したネットワークスループットの低下メカニズムに関する分析, 土木計画学研究・講演集, vol.52, 237, 秋田大学, November 2015.
\item 和田健太郎, 佐津川功季, 動的利用者均衡状態におけるMFDの解析: 1起点多終点ネットワークの場合, 土木計画学研究・講演集, vol.50, 125, 鳥取大学, November 2014.
\end{enumerate}

\section{競争的研究資金}
\begin{enumerate}[label={[\thesection.\arabic*]}]
\item 物流の脱炭素化に向けた電力と交通を融合した電気自動車のネットワーク解析理論の構築, 日本学術振興会: 科学研究費助成事業 基盤研究(B), 分担, 2025年4月–2028年3月, 分担金: 1500千円.
\item デジタルツイン時代の統合型輸送システムの分散型進化的最適化と実装, 日本学術振興会: 科学研究費助成事業 基盤研究(A), 分担, 2025年4月–2029年3月, 分担金: 200千円(現在までの実績).
\item 「つながる」「わけあう」交通システムを活かすための自己組織化理論, 鹿島学術振興財団: 国際共同研究, 分担, 2024年4月–2026年3月, 分担金: 600千円.
\item シェア型自動運転車交通システムの規範的数理モデルの開発とその展開, 日本学術振興会: 科学研究費助成事業 基盤研究(B), 分担, 2024年4月–2028年3月, 分担金: 700千円(現在までの実績).
\item マルチスケール時空間経済系における粒子・流体表現を統合したメカニズム設計理論, 日本学術振興会: 科学研究費助成事業 基盤研究(B), 分担, 2024年4月–2027年3月, 分担金: 1,000千円(現在までの実績).
\item 異質性がもたらす交通システムの進化特性解明とそれに基づく交通制御・計画の設計, 日本学術振興会: 科学研究費助成事業 若手研究, 代表, 2023年4月–2026年3月, 総額: 3,500千円.
\item 多様な都市活動を支援する予測情報共有型時空間リソース有効活用技術の研究開発, 国立研究開発法人 情報通信研究機構 (NICT): 高度通信・放送研究開発委託研究, 分担, 2021年4月–2023年3月, 分担金: 9,000千円(東北大学研究グループへの分担金).
\item 交通・物流システム効率化のための市場型マッチング・システムの設計・評価法構築, 日本学術振興会: 科学研究費助成事業 基盤研究(B), 分担, 2021年4月–2024年3月, 分担金: 1,500千円.
\item 交通工学理論と機械学習を融合した道路交通システムの状態推定・将来予測・制御, 日本学術振興会: 科学研究費助成事業 基盤研究(B), 分担, 2020年4月–2024年3月, 分担金: 2,400千円.
\item MaaS+CV時代の次世代交通システムに向けたインフラと制度の設計, 日本学術振興会: 科学研究費助成事業 基盤研究(A), 分担, 2020年4月–2024年3月, 分担金: 1,200千円.
\item 高度な動的制御のためのゲーム理論に基づくネットワーク交通流解析理論の構築, 日本学術振興会: 科学研究費助成事業 若手研究, 代表, 2020年4月–2023年3月, 総額: 2,800千円.
\item 社会便益を考慮した高速道路施設の維持管理高度化に関する研究, 日本学術振興会: 科学研究費助成事業 基盤研究(A), 分担, 2019年4月–2023年3月, 分担金: 2,500千円.
\item 交通渋滞の縮約表現に着目した大規模ネットワークの動的階層化による制御手法の研究, 日本学術振興会: 特別研究員奨励費 (DC2), 代表, 2018年4月–2020年3月, 総額: 1,700千円.
\end{enumerate}

\section{共同研究実績}
\begin{enumerate}[label={[\thesection.\arabic*]}]
\item 健康増進施策支援のための生活習慣病等の発症予測技術及び行動変容支援技術に関する研究, 日本電気株式会社, 2024年4月–2025年3月.
\item 道路設備のスマートメンテナンスに関する研究, 西日本高速道路エンジニアリング関西株式会社・中国株式会社・四国株式会社・九州株式会社,西日本高速道路ファシリティーズ株式会社, 2022年9月–現在.
\item 道路設備の予防保全に関する研究, 西日本高速道路エンジニアリング関西株式・西日本高速道路ファシリティーズ株式会社, 2018年4月–2022年9月.
\end{enumerate}

\section{社会貢献}
\subsection*{委員会等}
\begin{enumerate}[label={[\thesection.\arabic*]}]
\item 土木学会論文集 土木計画学(方法と技術)編集小委員会(第41小委員会), 土木学会, 編集委員, 2025年6月–現在.
\item 第2学術小委員会, 交通工学研究会, 委員兼幹事, 2025年4月–現在.
\item 東北地方研究会, 地域道路経済戦略研究会(国土交通省東北地方整備局), 委員, 2022年6月–現在.
\end{enumerate}
\subsection*{学会活動}
\begin{enumerate}[resume,label={[\thesection.\arabic*]}]
\item The 12th Triennial Symposium on Transportation Analysis (TRISTAN XII), TRISTAN XII, Session Chair, 2025.
\item The 12th Triennial Symposium on Transportation Analysis (TRISTAN XII), TRISTAN XII, Scientific Committee, 2024–2025.
\end{enumerate}
\subsection*{学術査読}
\noindent International Journal of Intelligent Transportation Systems Research; ITS Symposium; Scientific Reports; Transportation Research Interdisciplinary Perspectives; Transportation Research Part B: Methodological; Transportation Research Part C: Emerging Technologies; 交通工学研究発表会論文集; 土木学会論文集..\\
